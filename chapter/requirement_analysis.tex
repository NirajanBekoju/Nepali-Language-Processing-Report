\chapter{Requirement Analysis and Feasibility Study}
\section{Functional and Non Functional Requirements}
\subsection{Functional Requirements}
\begin{enumerate}[font=\bfseries]
    \item \textbf{R1: Word Embeddings Visualization}

          \textbf{Description :} Word embeddings of nepali words created using various methods like word2vec, embedding layers should be visualized in 2d and 3d graphs and show the required relationship between the words.

          \begin{enumerate}
              \item \textbf{Visualize word embeddings}

                    Input : Select word embeddings visualization tab

                    Output : 2d and 3d graphs of the word embeddings vector of few selected words or default 100 or 200 words(Number of words may vary as per requirements).

                    Processing : All the default word embeddings vectors will be returned and plotted in 2d and 3d graphs.

              \item \textbf{Search Words}

                    Input : Keyword to be searched in the graph

                    Output: Vector of keyword and position in the graphs highlighted

                    Processing: Keyword searched in the vocabulary and if found its vector is returned and plotted with highlight color in the graph. If not found, display NOT FOUND message. If the keyword is not found in the vocabulary, archive it to the vocabulary database for future references.
          \end{enumerate}

    \item \textbf{R2: Sentiment Classification}

          \textbf{Description : }When the user enters a sentence and performs sentimental classification, the sentimental classification should be able to distinguish the sentence into positive, negative and neutral with some cutoff probability (about 70\%).

          \begin{enumerate}
              \item \textbf{Classify and result}

                    Input : A sentence

                    Ouput : Probabilities of the sentence being positive, negative and neutral.

                    Processing :

              \item \textbf{Store the result}

                    Input : A sentence to be classified

                    Ouput : Response message for successful storage

                    Processing : Store the sentence and its predicted label in the sentimental classification table. If the predicted label is incorrect , get the correct label from the user and store it in the database.
          \end{enumerate}

    \item \textbf{R3: Language Modelling}

          \textbf{Description : }Nepali Language modeling based on both probabilistic and neural approach. Based on the trained language model, it should be able to auto generate the next word given context words.

          \begin{enumerate}
              \item \textbf{Predict next word}

                    Input : An incomplete sentence

                    Ouput : List of words along the probabilities of being the next word.

                    Processing : The language model should be able to predict the next word from vocabulary based on the previous words provided by the user.
          \end{enumerate}

    \item \textbf{R4: Spelling Correction}

          \textbf{Description : }Based on the trained language model, the system should suggest correct spelling for a given sentence.

          Input : A sentence

          Ouput : Provide suggestions to correct the spelling of the words in the input sentence.

          Processing : Firstly, candidate sentences are found using  1 or 2 minimum edit distance. Probability of each candidate sentence and likelihood of error of given sentence is calculated. And suggestions should be made based on maximum posterior.

    \item \textbf{R5: Interactive Web Application}
    
          \textbf{Description :} All the requirements R1, R2, R3 and R4 should be easily accessible through a website and APIs.
\end{enumerate}

\subsection{Non Functional Requirements}
\begin{enumerate}[font=\bfseries]
    \item \textbf{R1: Performance and scalability}
          \begin{enumerate}
              \item The load time for the user interface screens shall take no longer than 3 seconds.
              \item Queries shall return results within 3 seconds.
          \end{enumerate}
    \item \textbf{R2: Design Constraints}
          \begin{enumerate}
              \item The system shall be developed using python and postgresql databases.
          \end{enumerate}
    \item \textbf{R3: Standard Compliance}
          \begin{enumerate}
              \item The graphical user interface shall have a consistent look and feel.
          \end{enumerate}
    \item \textbf{R4: Availability}
          \begin{enumerate}
              \item The system shall be available all time.
          \end{enumerate}
    \item \textbf{R5: Portability and Compatibility}
          \begin{enumerate}
              \item The system shall be able to run in all browsers.
          \end{enumerate}
    \item \textbf{R6: Reliability and Maintainability}
          \begin{enumerate}
              \item The mean time to recover from a system failure must not be greater than 2 hours.
              \item Rate of system failure should not be greater than twice a month.
          \end{enumerate}
\end{enumerate}

\section{Hardware Requirements}
\begin{enumerate}
    \item GeForce RTX 3060 for prototype development 
    \item Minimum 32 GB RAM 
    \item Minimum 15 GB Storage
\end{enumerate}

\section{Software Requirements}
\begin{enumerate}
    \item Programming Language : Python, Javascript
    \item Libraries : Tensorflow, keras, scikit-learn, spacy, nltk, and other python libraries 
    \item Framewoks : Django, Django Rest Framework, React
    \item Application software : Postman
    \item Target Platform : Web 
\end{enumerate}

\section{Feasibility Study}
Feasibility is defined as the practical extent to which a project can be performed successfully. To evaluate feasibility, a feasibility study is performed, which determines whether the solution considered to accomplish the requirements is practical and workable in the software.


\subsection{Technical Feasibility}
Technical Feasibility is the formal process of assessing whether it is technically possible to manufacture a product or service. For our system, the required software and hardware were readily available to us. So we can state with confidence that the project is technically feasible.

\subsection{Economic Feasibility}
The purpose of an Economic Feasibility Study (EFS) is to demonstrate the net benefit of a proposed project, taking into consideration the benefits and costs to the agency, other state agencies, and the general public as a whole. It is the  process of determining whether the project  is worth the cost and time investment. Since most of the hardware was already in our possession and the softwares used was mostly free, we can declare that the system is economically feasible.

\subsection{Legal Feasibility}
Since we are using open source data for building language model, we can state that the project is legally feasible.

\subsection{Scheduling Feasibility}
This assessment is the most important for project success; after all, a project will fail if not completed on time. In scheduling feasibility, an organization estimates how much time the project will take to complete. Since we have properly planned our approach to completing the project in components using agile model, we can say that it is feasible timewise.
